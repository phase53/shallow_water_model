\documentclass[dvipdfmx,uplatex]{jsarticle}
\usepackage{graphicx}
\usepackage{amsmath,amssymb}
\usepackage{amsthm}
\usepackage[binary-units=trues]{siunitx}
\usepackage{latexsym}
\usepackage{ulem}
\usepackage{bm}
\usepackage{here}
\usepackage{multirow}
\usepackage{tabularx}
\usepackage{physics}
\usepackage{color}
 \usepackage[many]{tcolorbox}
 \usepackage{adjustbox}
 \usepackage {tablefootnote}
 \pagestyle{headings}
\newcolumntype{C}[1]{>{\centering\arraybackslash}p{#1}}
\newcolumntype{L}[1]{>{\raggedright\arraybackslash}p{#1}}
\newcolumntype{R}[1]{>{\raggedleft\arraybackslash}p{#1}}
\renewcommand{\Re}{\operatorname{Re}}
\renewcommand{\Im}{\operatorname{Im}}
\makeatletter
\renewcommand{\@cite}[1]{\textsuperscript{#1)}}
\renewcommand{\@biblabel}[1]{#1)}
 \renewcommand{\thefootnote}{\arabic{footnote}}
\makeatother

\makeatletter
\renewcommand\paragraph{\@startsection{paragraph}{4}{\z@}%
  {-3.25ex\@plus -1ex \@minus -.2ex}%
  {1.5ex \@plus .2ex}%
  {\normalfont\normalsize\bfseries}}
\makeatother

\title{}
\author{}
\date{}
\begin{document}
%\maketitle
\section{支配方程式}
支配方程式は以下のとおりである.
\begin{eqnarray}
\label{zeta} \cfrac{\partial q}{\partial t}&=&-\cfrac{\partial (uq)}{\cos{\phi}\partial\lambda}-\cfrac{\partial (vq\cos{\phi})}{\cos{\phi}\partial\phi} \\
\label{div} \cfrac{\partial D}{\partial t}&=&\cfrac{\partial (vq)}{\cos{\phi}\partial\lambda}-\cfrac{\partial (uq\cos{\phi})}{\cos{\phi}\partial\phi}-\nabla^2(E+h) \\
\label{depth} \cfrac{\partial h}{\partial t}&=&-\cfrac{\partial (hu)}{\cos{\phi}\partial\lambda}-\cfrac{\partial (hv\cos{\phi})}{\cos{\phi}\partial\phi} 
\end{eqnarray}
ここで,各記号は次の意味を持つ.
\begin{table}[h]\centering
\begin{tabular}{cc}
記号 & 変数  \\ \hline
$\lambda$ & 経度 \\
$\phi$ & 緯度 \\
$t$ & 時間 \\
$u$ & 経度方向の流速 \\
$v$ & 緯度方向の流速 \\
$D$ & 発散 \\
$q$ & 相対渦度 \\
 $h$ &  水深 \\ \hline
\end{tabular}
\end{table}
\section{モデルの設定・使い方}
\subsection{前提}
本モデルはISPACK-3.2.2を使用しているため,Makefile内のLIBSに,コンパイルしたISPACKの場所を指定してあげる必要がある.コンパイラの指定は同ファイル内のFCで行うが,gfortran以外の動作は未検証である.またopenMPも前提としている.\\
\indent 波数空間への変換は三角切断を前提としている.時間積分には散逸項を除いて4次のRunge-Kutta法を用いていて,散逸項は解核行列を用いて計算している.\\
\indent 本モデルをそのまま用いる場合は散逸項として粘性項(式(\ref{zeta}),(\ref{div}),(\ref{depth})の右辺に加わる)が計算に含まれている,必要ない場合はmain.f90に含まれるサブルーチンMKRSLVにおいて,配列RSLVNTの全成分に$1.\_ 8$を入れればよい.
\subsection{各変数について}
その他,モデルの設定の中でユーザーが操作する必要があるものは全てf90/main.f90最上部のグローバル変数モジュール内にまとまっている.下の表ははそれらのうち,特に変更を要するものを挙げたものである.
\begin{table}[h]\centering
\begin{tabular}{ccc}
型 & 変数名 & 意味  \\ \hline
CHARACTER(128) & DATPAS & データディレクトリパスの場所 \\
CHARACTER(128) & FNZETS & 渦度場の出力ファイルの名前  \\
CHARACTER(128) & FNDIVS & 発散場のの出力ファイルの名前  \\
CHARACTER(128) & FNDPHS & 水深場の出力ファイルの名前  \\
CHARACTER(128) & INZETS & 渦度場の入力ファイルの名前  \\
CHARACTER(128) & INDIVS & 発散場のの入力ファイルの名前  \\
CHARACTER(128) & INDPHS & 水深場の入力ファイルの名前  \\
LOGICAL & RD & 入力ファイルを設定したとき.TRUE.とする\tablefootnote{.FALSE.の場合は自動的のモデル内のMKINITによって与えられるガウシアン状の水深の場が初期値になる}  \\
INTEGER(8) & SSTEP & 計算の開始日\\
INTEGER(8) & JM & 緯度方向の格子点数 \\
INTEGER(8) & IM & 経度方向の格子点数 \\
INTEGER(8) & NN & 切断波数 \\
INTEGER(8) & DAY & 計算日数 \\
INTEGER(8) & DAYSTP & 1日あたりの計算ステップ数\tablefootnote{時間ステップ幅はこれの逆数になるので,CFL条件に注意} \\
REAL(8) & VSCSTY & 粘性係数\tablefootnote{変える必要があるのは時定数(デフォルトは1/10なのでその逆数で分子に10がかかっている部分)} \\
\end{tabular}
\end{table}
また,出力されるファイルは波数空間(つまり,球面調和関数展開の展開係数が入っている.配列の並びなどはISPACKマニュアルを参照)のものである,これを図示などのために物理空間に変換するには,別ファイルとして用意されているconv.f90を用いれば良いが,conv.f90内にも上の表の変数がいくつか含まれるため注意.入力するファイルもISPACKに準拠した波数空間のデータである必要があり,作成にはISPACKを用いるのが早いと思う.\\
\indent これらを設定した後は,f90ディレクトリ内でmakeし./main と入力すれば積分が始まる.出力される各データファイル名と切断波数などの情報が最初に標準出力され,その後現在計算中のステップを次々標準出力する.DAYSTPで定めたステップごと(つまり,計算日数で数えて1日ごと)に各変数についてファイルの出力が行われる.よって,出力ファイル数は$3\times$DAY個になるはずである.
\subsection{図示}
図示のためにpy/plot$\_$main.pyが用意されている.これを使う場合matplotlibとnumpyが別途必要である.また,図示用のカラーマップとして同ディレクトリ内にcolormap.txtが入っている.これは,Additional colormaps for DCL(https://www.gfd-dennou.org/arch/koshiro/comp/dcl/clrmap/)のcolormap$\_$67から作成した.
\section{今後の修正予定}
\begin{itemize}
\item このマニュアルを詳しくかく,
\item 時定数を分離して設定しやすいようにしておく.
\item 各モジュールを別ファイルにする.コンパイラ依存性など確認できてないところをチェック.
\item 解析などで便利なようにnamelistを作成してパラメタを一括で管理できるようにする.
\item RHSの高速化,計算のホットスポットなので優先,openMPを非線形項に適用する.
\end{itemize}
\end{document} 
